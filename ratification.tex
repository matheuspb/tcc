\documentclass[11pt]{letter}

\usepackage[brazil]{babel}
\usepackage[T1]{fontenc}
\usepackage[a4paper, margin=1.5cm]{geometry}
\usepackage[colorlinks, urlcolor=blue, citecolor=red]{hyperref}
\usepackage[utf8]{inputenc}

\usepackage[table]{xcolor}
\usepackage{mathptmx, multirow, hhline}

\begin{document}

\pagestyle{empty}

\begin{center}
\textbf{
	DEPARTAMENTO DE INFORMÁTICA E ESTATÍSTICA --- CTC --- UFSC \\
	RATIFICAÇÃO DE PLANO DE TRABALHO DO SEMESTRE PARA DESENVOLVIMENTO DE TCC
}
\end{center}

\vspace{1em}
\setlength\extrarowheight{5pt}
\begin{tabular}{l l}
	\textbf{Disciplina:} & INE5434 -- Trabalho de Conclusão de Curso II     \\
	\textbf{Curso:}      & Ciência da Computação                            \\
	\textbf{Autor:}      & Matheus Silva Pinheiro Bittencourt               \\
	\textbf{Título:}     & Reducing keys in Rainbow-like signature schemes  \\
	\textbf{Professor responsável:} & Ricardo Felipe Custódio               \\
\end{tabular}


\vspace{1em}
{\large \textbf{Objetivos}}

\textit{General goal:} Study and describe Rainbow-like digital signature
schemes, understand the optimizations that reduce keys and signature sizes, and
their impact on the security of the classic schemes. Observe and analyze the
impact of parameter selection for such algorithms, as this plays an important
role in efficient and fast implementations of the schemes. Analyze the
state-of-the-art schemes, to understand the strategies being used to optimize
the cryptosystems.

\textit{Specific goals:} Describe the classic OV and the UOV digital signature
schemes; Describe the Rainbow signature scheme; Introduce relevant
optimizations on Rainbow, like CyclicRainbow; Compare and analyze the
performance of the aforementioned schemes in terms of operations needed to
generate and verify signatures as well as storage requirements. Finally,
propose new optimizations on top of those schemes.

\vspace{1em}
{\large \textbf{Cronograma}}

\begin{center}
	\begin{tabular}{|p{4.04cm}|*{12}{c|}}
		\hline & \multicolumn{6}{c|}{\textbf{2019}}
			& \multicolumn{6}{c|}{\textbf{2019}} \\ \cline{2-13}
		\multicolumn{1}{|c|}{\multirow{-2}{*}{\textbf{Etapas}}}
			& \textbf{jan.} & \textbf{fev.} & \textbf{mar.}
			& \textbf{abr.} & \textbf{mai.} & \textbf{jun.}
			& \textbf{jul.} & \textbf{ago.} & \textbf{set.}
			& \textbf{out.} & \textbf{nov.} & \textbf{dez.} \\
		\hline Fundamentação teórica
			& \cellcolor{lightgray} & & & & & & & & & & & \\
		\hline Revisão do estado da arte
			& \cellcolor{lightgray} & \cellcolor{lightgray}
			& \cellcolor{lightgray} & & & & & & & & & \\
			\hline Desenvolvimento do TCC
			& & & \cellcolor{lightgray} & \cellcolor{lightgray}
			& \cellcolor{lightgray} & & & & & & & \\
		\hline Implementação
			& & & & & \cellcolor{lightgray} & \cellcolor{lightgray}
			& & & & & & \\
		\hline Experimentação
			& & & & & & & \cellcolor{lightgray} & & & & & \\
		\hline Implementação
			& & & & & & & & \cellcolor{lightgray} & & & & \\
		\hline Redação da monografia
			& & & & & & & & \cellcolor{lightgray} & \cellcolor{lightgray}
			& \cellcolor{lightgray} & & \\
		\hline Ajustes na monografia
			& & & & & & & & & & \cellcolor{lightgray} & \cellcolor{lightgray}
			& \\
		\hline Defesa pública
			& & & & & & & & & & & \cellcolor{lightgray} & \\
		\hline Ajustes finais do TCC
			& & & & & & & & & & & & \cellcolor{lightgray} \\
		\hline
	\end{tabular}

	\vspace*{\fill}
	\fbox{
		\begin{minipage}[c][6em][c]{0.7\textwidth}{
			\center\textbf{Preenchimento pelo professor responsável pelo TCC} \\
			}
			\vspace{3mm}
			\qquad $(\quad)$ \ Ciente e de acordo.
			\qquad Assinatura:

			\qquad Data: \_\_ / \_\_ / \_\_\_\_
		\end{minipage}
	}
\end{center}

\end{document}
