\documentclass{ufsctex/ufsctex}

\usepackage[table]{xcolor}
\usepackage{mdframed, enumitem, multirow, hhline, amssymb}

\newcolumntype{P}[1]{>{\centering\arraybackslash}m{#1}}

\newcommand{\thatcell}[3]{
  \multicolumn{#1}{#2}{\cellcolor{lightgray} \textbf{#3}}
}

\begin{document}

\instituicao[a]{Universidade Federal de Santa Catarina}
\departamento[o]{Departamento de Informática e Estatística}
\curso[o]{Programa de Graduação em Ciência da Computação}
\documento[a]{Monografia}
\titulo{Multivariate Public-Key Cryptosystems}
\autor{Matheus Silva Pinheiro Bittencourt}
\grau{Bacharel em Ciência da Computação}
\local{Florianópolis}
\data{12}{Novembro}{2018}
\orientador[Orientador]{Prof.\ Dr.\ Ricardo Felipe Custódio}
\coorientador[Coorientador]{Gustavo Zambonin}
\coordenador[Coordenador]{Prof.\ Dr.\ Rafael Luiz Cancian}

\textoResumo{
Os algoritmos clássicos de assinatura digital como RSA e ECDSA baseiam sua
segurança na dificuldade de fatorar números inteiros, e no logaritmo discreto,
respectivamente. Esses problemas já possuem algoritmos quânticos que os
resolvem em tempo polinomial, ou seja, com computadores quânticos poderosos o
suficiente, o uso dos algoritmos de assinatura digital mais difundidos se
tornará impraticável. Naturalmente, com o desenvolvimento de computadores
quânticos cada vez mais poderosos, o interesse por criptossistemas resistentes
a ataques de computadores quânticos também cresceu, surgindo uma nova área
chamada criptografia pós-quântica que visa desenvolvê-los. Os algoritmos de
criptografia pós-quânticos se baseiam numa série de problemas que ainda se
mantém difíceis mesmo que computadores quânticos poderosos estejam disponíveis,
logo, despertam o interesse para o uso nesse cenário. Este trabalho visa
estudar a classe de criptossistemas baseados em polinômios multivariados, que
se baseiam em problemas como o \textit{Polynomial System Solving} e o
\textit{Isomorphism of Polynomials}, que ainda são difíceis mesmo com
algoritmos quânticos, portanto são bons candidatos para criptografia
pós-quântica.
}
\palavrasChave{criptografia, assinatura digital, pós-quântico}

\capa{}
\folhaderosto{}

\clearpage

\begin{mdframed}[backgroundcolor=lightgray, linewidth=0pt]
    \centering
    \textbf{FOLHA DE APROVAÇÃO DE PROPOSTA DE TCC}
\end{mdframed}

\vspace{-5mm}
\begin{table}[h]
  \begin{tabular}{|>{\bfseries}l|p{6.57cm}|}
    \hline
    Acadêmico(s)            & Matheus Bittencourt \\ \hline
    Título do trabalho      & Multivariate Public Key Cryptosystems \\ \hline
    Curso                   & Ciência da Computação/INE/UFSC        \\ \hline
    Área de Concentração    & Matemática da Computação              \\ \hline
  \end{tabular}
\end{table}

\vspace{-2mm}
{\footnotesize\noindent\textbf{
  Instruções para preenchimento pelo \underline{ORIENTADOR DO TRABALHO}:} \\
  \begin{itemize}[leftmargin=3.6mm,label=-]
    \vspace{-4mm}
    \item Para cada critério avaliado, assinale um X na coluna SIM apenas
        se considerado aprovado. Caso contrário, indique as alterações
        necessárias na coluna Observação.
  \end{itemize}
}

\vspace{-3mm}
\begin{table}[hbpt]
  \begin{tabular}{|>{\tiny}m{4.3cm}*{4}{|>{\columncolor{lightgray}\tiny}c}|c|}
    \hline
    \rowcolor{lightgray} & \multicolumn{4}{c|}{\textbf{Aprovado}} & \\
    \hhline{|>{\arrayrulecolor{lightgray}}->{\arrayrulecolor{black}}|
        |---->{\arrayrulecolor{lightgray}}->{\arrayrulecolor{black}}|}
    \rowcolor{lightgray}
    \multicolumn{1}{|c|}{\multirow{-2}{*}{\normalsize\textbf{Critérios}}}
      & \textbf{Sim} & \textbf{Parcial} & \textbf{Não} & \textbf{Não se aplica}
      & \multirow{-2}{*}{\textbf{Observação}} \\ \hline
    1. O trabalho é adequado para um TCC no CCO/SIN
      (relevância / abrangência)?                         & & & & & \\ \hline
    2. O título do trabalho é adequado?                   & & & & & \\ \hline
    3. O tema de pesquisa está claramente descrito?       & & & & & \\ \hline
    4. O problema/hipóteses de pesquisa do trabalho
      está claramente identificado?                       & & & & & \\ \hline
    5. A relevância da pesquisa é justificada?            & & & & & \\ \hline
    6. Os objetivos descrevem completa e claramente
      o que se pretende alcançar neste trabalho?          & & & & & \\ \hline
    7. É definido o método a ser adotado no trabalho?
      O método condiz com os objetivos e é adequado
      para um TCC?                                        & & & & & \\ \hline
    8. Foi definido um cronograma coerente com o método
      definido (indicando todas as atividades) e com as
      datas das entregas (p.ex. Projeto I, II, Defesa)?   & & & & & \\ \hline
    9. Foram identificados custos relativos à execução
      deste trabalho (se houver)? Haverá financiamento
      para estes custos?                                  & & & & & \\ \hline
    10. Foram identificados todos os envolvidos neste
      trabalho?                                           & & & & & \\ \hline
    11. As formas de comunicação foram definidas
      (ex.: horários para orientação)?                    & & & & & \\ \hline
    12. Riscos potenciais que podem causar desvios do
      plano foram identificados?                          & & & & & \\ \hline
    13. Caso o TCC envolva a produção de um software ou
      outro tipo de produto e seja desenvolvido também
      como uma atividade realizada numa empresa ou
      laboratório, consta da proposta uma declaração
      (Anexo 3) de ciência e concordância com a entrega
      do código fonte e/ou documentação produzidos?       & & & & & \\ \hline
  \end{tabular}

  \vspace{2mm}
  {\footnotesize
  \begin{tabular}{|>{\bfseries}p{3cm}|l|l|l|}
    \hline Avaliação & \multicolumn{2}{l}{\bf $\square$ Aprovado}
      & \textbf{$\square$ Não Aprovado} \\
    \hline Professor Responsável & Ricardo Felipe Custódio & 12/11/2018 & \\
    \hline
  \end{tabular}}
\end{table}

\paginaresumo

\sumario

\chapter{Introduction}

Classic cryptography is threatened by quantum computers since polynomial time
quantum algorithms for solving the factorization and the discrete logarithm
problem already exist~\cite{shor1999polynomial}. With such algorithms, ECDSA
and RSA private keys could be easily recovered with a sufficiently powerful
quantum computer, hence the most used digital signature algorithms would become
insecure and not suitable for signing digital documents.

In such scenario, cryptosystems that can't be broken by quantum computers
should be used for generating new digital signatures. The area that studies
such cryptosystems is called post-quantum cryptography, and the interest in
these has emerged with the development of quantum computers.  These algorithms
base their security upon problems that still do not have polynomial time
quantum algorithms that solves them, hence are good candidates for use in a
scenario with attackers equipped with quantum computers.

There are several classes of post-quantum cryptosystems, and each of them
relies on one kind of problem, this work aims to study the class of
cryptosystems called Multivariate Public-Key Cryptosystems (MPKCs). These
cryptosystems are constructed using multivariate polynomials systems, and
differ from the classic polynomial systems on the fact that the monomials can,
and will be, a product of more than one variable. Solving simple polynomial
systems is trivial using Gaussian elimination, but solving multivariate systems
is proven to be NP-Hard~\cite{garey1990npc} and the security of MPKCs will
mostly rely on this fact and on the hardness of the Isomorphism of Polynomials
(IP) problem.

A lot of digital signature schemes were developed upon the structure of
multivariate systems, one of the first schemes presented was the Oil-Vinegar
(OV) signature scheme~\cite{patarin1997ov}, which is currently
broken~\cite{kipnis1998cryptanalysis} in its original specification, but a
subsequent work made it secure again with the correct
parameters~\cite{kipnis1999unbalanced} leading to a scheme called Unbalanced
Oil-Vinegar (UOV). The classic OV is the building block for various other
schemes, called \textit{SingleField} schemes, that, in essence, are similar to
OV, but ended up optimizing the security, the signature size and the key sizes.
These schemes withstood the test of time and the UOV is still secure.

This work aims to understand the OV signature scheme and its subsequent
optimized schemes as well as propose new optimization(s). Reducing public parts
of the systems as well as the signatures size is very interesting, because
these are the information that are actually transmitted over a communication
channel and stored in various places, like embedded systems, in which these
resources may be expensive. Also reducing key sizes has a direct impact on the
operations that use them, from the simple fact that we operate with less data.
For example, with a smaller public key, verifying signatures becomes faster, we
can see that in a work by Petzold~\textit{et al.}~\cite{petzoldt2011small}.

\section{Goals}

This section describes general and specific goals of the proposed work as well
as scope and other basic premises.

\begin{itemize}

    \item \textit{General goals:} Study and describe UOV-like digital
    signatures schemes, understand the optimized schemes that reduce keys and
    signature sizes, and their impact on the security of the classic schemes.
    Observe and understand the impact of the parameters selection for such
    schemes and experiment with them, as this plays an important role in
    efficient and fast implementations of the schemes. Analyze the bleeding
    edge schemes recently proposed, searching for pitfalls and also understand
    the strategies being used to optimize the cryptosystems.

    \item \textit{Specific goals:} Describe the classic OV and the UOV digital
    signature schemes; describe the Rainbow signature scheme; describe the
    relevant optimized schemes based on Rainbow like
    CyclicRainbow~\cite{petzoldt2010cyclicrainbow}; Compare and analyze the
    performance of the aforementioned schemes in terms of time to perform
    signatures and verifications as well as memory usage. If possible propose
    new optimizations on top of those schemes.

    \item \textit{Scope:} This work will not cover other classes of
    post-quantum algorithms like code-based, lattice-based and hash-based
    cryptosystems, also, classic asymmetric algorithms (RSA, (EC)DSA, etc.)
    won't be covered. Quantum algorithms and their aspects will not be covered
    by this work, it will be restricted to algorithms that run on classic
    computers, that is, computers based on binary digital systems.

    \item \textit{Acceptance criteria:} Detailed study and implementation of
    the (U)OV and Rainbow signature schemes; Detailed study of at least three
    optimized schemes and the impact caused by the respective optimizations.

    \item \textit{Deliverables:} Reports of the undergraduate thesis according
    to the deadlines and specifications of the related UFSC disciplines,
    including the full thesis at the end of the course (INE5434), as well as
    the documented implementations made for experimentation and testing.

    \item \textit{Premises and restrictions:} There will be periodic meetings
    with the supervisors to discuss what is being researched and what should be
    researched. All documents will be normalized according to UFSC thesis
    rules. Free and open-source software will be used. All deadlines of the
    project, in respect of the related disciplines, should be respected.

\end{itemize}

\section{Methodology}

The work will be developed using the infrastructure and resources provided by
the Computer Security Laboratory (LabSEC/UFSC). A literature review will be
made to determine what is the state-of-the-art in MPKCs, recently proposed
schemes will be studied as well as broken ones for a better understanding of
the used constructions that optimize the classic multivariate schemes. Works
that investigate the parameters of SingleField schemes will be given the
deserved attention as well. Performance of all the schemes studied should be
observed along with the impact of the optimizations.

\chapter{Cronograma}

\begin{figure}[htbp]
  \begin{tabular}{|p{4.04cm}|*{6}{c|}}
    \hline \rowcolor{lightgray}
      & \multicolumn{6}{c|}{\textbf{Meses -- 2019}} \\
    \hhline{|>{\arrayrulecolor{lightgray}}->{\arrayrulecolor{black}}|
      |------>{\arrayrulecolor{lightgray}}>{\arrayrulecolor{black}}|}
    \rowcolor{lightgray}
      \multicolumn{1}{|c|}{\multirow{-2}{*}{\textbf{Etapas}}}
      & \textbf{jan.} & \textbf{fev.} & \textbf{mar.}
      & \textbf{abr.} & \textbf{mai.} & \textbf{jun.} \\
    \hline Fundamentação teórica & \cellcolor{lightgray} & & & & & \\
    \hline Revisão do estado da arte & \cellcolor{lightgray}
      & \cellcolor{lightgray} & & & & \\
      \hline Desenvolvimento do TCC & & \cellcolor{lightgray}
      & \cellcolor{lightgray} & \cellcolor{lightgray} & & \\
    \hline Implementação & & & & \cellcolor{lightgray}
      & \cellcolor{lightgray} & \cellcolor{lightgray} \\
    \hline Relatório de TCC I & & & & & \cellcolor{lightgray} & \\
    \hline
    \hline \rowcolor{lightgray}
      & \multicolumn{6}{c|}{\textbf{Meses -- 2019}} \\
    \hhline{|>{\arrayrulecolor{lightgray}}->{\arrayrulecolor{black}}|
      |------>{\arrayrulecolor{lightgray}}>{\arrayrulecolor{black}}|}
    \rowcolor{lightgray}
      \multicolumn{1}{|c|}{\multirow{-2}{*}{\textbf{Etapas}}}
      & \textbf{jul.} & \textbf{ago.} & \textbf{set.}
      & \textbf{out.} & \textbf{nov.} & \textbf{dez.} \\
    \hline Ajustes na implementação & \cellcolor{lightgray} & & & & & \\
    \hline Redação da monografia & \cellcolor{lightgray}
      & \cellcolor{lightgray} & \cellcolor{lightgray} & & & \\
    \hline Ajustes na monografia & & & \cellcolor{lightgray}
      & \cellcolor{lightgray} & & \\
    \hline Relatório de TCC II & & & & & \cellcolor{lightgray} & \\
    \hline Defesa pública & & & & & & \cellcolor{lightgray} \\
    \hline Ajustes finais do TCC & & & & & & \cellcolor{lightgray} \\
    \hline
  \end{tabular}
\end{figure}

\chapter{Custos}

\begin{figure}[htbp]
  \begin{tabular}{|p{1.69cm}|*{3}{l|}}
    \hline \thatcell{1}{|c|}{Item} & \thatcell{1}{c|}{Quantidade}
      & \thatcell{1}{c|}{Valor unitário (R\$)}
      & \thatcell{1}{c|}{Valor Total (R\$)}                         \\
    \hline \thatcell{4}{|l|}{Material permanente}                   \\
    \hline Computador   & 1     & R\$ 3.000,00  & R\$ 3.000,00      \\
    \hline Internet     & 1     & R\$ 1.000,00  & R\$ 1.000,00      \\
    \hline Artigos      & 10    & R\$ 90,00     & R\$ 900,00        \\
    \hline Livros       & 2     & R\$ 200,00    & R\$ 400,00        \\
    \hline \thatcell{4}{|l|}{Material de consumo}                   \\
    \hline Alimentação  & 264   & R\$ 10,00     & R\$ 2.640,00      \\
    \hline CDs/DVDs     & 4     & R\$ 2,00      & R\$ 8,00          \\
    \hline \thatcell{4}{|l|}{Outros recursos e serviços}            \\
    \hline Impressões   & 200   & R\$ 1,00      & R\$ 200,00        \\
    \hline
  \end{tabular}
\end{figure}

\chapter{Recursos Humanos}

\begin{figure}[htbp]
  \begin{tabular}{|*{2}{p{4.96cm}|}}
    \hline \rowcolor{lightgray}
    \thatcell{1}{|c|}{Nome}                   & \thatcell{1}{c|}{Função} \\
    \hline Matheus Silva Pinheiro Bittencourt & Autor                    \\
    \hline Ricardo F. Custódio                & Orientador               \\
    \hline Gustavo Zambonin                   & Coorientador             \\
    \hline Renato Cislaghi                    & Coordenador de projetos  \\
    \hline Lucas Pandolfo Perin               & Membro da banca          \\
    \hline
  \end{tabular}
\end{figure}

\chapter{Comunicação}

\begin{figure}[htbp]
  \footnotesize
  \begin{tabular}{|P{1.9cm}|P{0.7cm}|P{2.5cm}|P{1.5cm}|P{2cm}|}
    \hline \rowcolor{lightgray}
    \textbf{O que precisa ser comunicado} & \textbf{Por quem}
      & \textbf{Para quem} & \textbf{Melhor forma de comunicação}
      & \textbf{Quando e com que frequência} \\
    \hline Enviar plano de projeto & Autor
      & Orientador, coorientador, coordenador de projetos & Sistema de TCC
      & Única vez, até dia 12/11/2018 \\
    \hline Entrega de relatório de TCC I & Autor
      & Orientador, coorientador, coordenador de projetos,
      membro(s) da banca & Sistema de TCC
      & Única vez, ao final do semestre 2019/1 \\
    \hline Entrega de relatório de TCC II & Autor
      & Orientador, coorientador, coordenador de projetos,
      membros(s) da banca & Sistema de TCC
      & Única vez, em meados do semestre 2019/2 \\
    \hline Defesa do TCC & Autor
      & Orientador, coorientador, coordenador de projetos,
      membro(s) da banca & Pessoalmente
      & Única vez, em meados do semestre 2019/2 \\
    \hline Entrega final da monografia & Autor
      & Orientador, coorientador, coordenador de projetos,
      membro(s) da banca & Sistema de TCC
      & Única vez, após a defesa, antes do término de 2019/2 \\
    \hline Reuniões de acompanhamento da pesquisa & Autor
      & Orientador, coorientador & Pessoalmente, webconferência
      & Quinzenalmente \\
    \hline Monitoramento do projeto & Autor
      & Orientador, coorientador & E-mail & Esporadicamente \\
    \hline
  \end{tabular}
\end{figure}

\chapter{Riscos}

\begin{figure}[htbp]
  \footnotesize
    \begin{tabular}{|P{1.3cm}|*{3}{P{0.8cm}|}*{2}{P{2.24cm}|}}
    \hline \rowcolor{lightgray}
    \textbf{Risco} & \textbf{Proba\-bilidade} & \textbf{Impacto}
      & \textbf{Priori\-dade} & \textbf{Estratégia de resposta}
      & \textbf{Ações de prevenção} \\
    \hline Paralisação de transporte público & Média & Médio & Baixa
      & Transportar-se à Universidade utilizando meios alternativos
      & Combinar transporte alternativo \\
    \hline Paralisação de servidores públicos & Muito baixa & Alto & Média
      & Produzir monografia e pesquisa utilizando recursos pessoais
      & Não se aplica \\
    \hline Problemas de saúde & Baixa & Alto & Alta
      & Tratamento médico das condições identificadas
      & Diminuição do fator de exposição em caso de doenças com fator
        ambiental, e exames para verificar condições genéticas \\
    \hline Perda de dados & Muito baixa & Alto & Média
      & Recuperar cópia de segurança
      & Cópias de segurança periódicas do material produzido \\
    \hline Queima de equipamento(s) eletrônico(s) & Muito baixa & Alto & Média
      & Comprar novo(s) equipamento(s)
      & Evitar utilização do(s) equipamento(s) em más condições de tempo ou
        por períodos muito prolongados \\
    \hline
  \end{tabular}
\end{figure}

\bibliographystyle{abnt-alf}
\bibliography{ref}

\end{document}
