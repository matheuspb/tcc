\documentclass[]{beamer}

\usepackage[T1]{fontenc}
\usepackage[utf8]{inputenc}
\usepackage[brazil]{babel}
\usepackage{amsmath, amsthm, makecell, tikz}

\title{Reducing keys in Rainbow-like signature schemes}

\author{
	Matheus Silva Pinheiro Bittencourt\inst{1}
}

\institute{
	\inst{1}
	\texttt{\href{mailto:matheus.spb@grad.ufsc.br}{matheus.spb@grad.ufsc.br}}\\
	Universidade Federal de Santa Catarina\\
	Departamento de Informática e Estatística
}

\date{11 de novembro de 2019}

\begin{document}

\frame{\titlepage}

\begin{frame}{Assinaturas Digitais}
	\begin{itemize}[<+->]
		\item Simulam as assinaturas de punho em meios digitais.
		\item Utilizam pares de chaves criptográficas, sendo que somente o
		assinante possui a chave privada.
		\item São verificadas a partir de um algoritmo público de verificação.
		\item Deve ser difícil criar novas assinaturas sem a chave privada.
		\item Provê integridade, autenticidade e não-repúdio dos documentos
		assinados.
		\item Os algoritmos mais utilizados são o RSA e o ECDSA.
	\end{itemize}
\end{frame}

\begin{frame}{Criptografia pós-quântica}
	\begin{itemize}[<+->]
		\item Os algoritmos clássicos de assinatura digital se baseiam em
		problemas da teoria dos números, como a fatoração de inteiros e o
		logaritmo discreto.
		\item Com o advento da computação quântica, esses problemas poderão ser
		resolvidos em tempo polinomial~\cite{shor1999polynomial}.
		\item Motiva o estudo de novos algoritmos criptográficos que se baseiam
		em outros problemas.
		\item Essa área de estudos é denominada criptografia pós-quântica.
		\item Estes algoritmos estão em processo de padronização pelo NIST.
		\item Este trabalho visa estudar uma classe de algoritmos de assinatura
		digital pós-quânticos.
	\end{itemize}
\end{frame}

\begin{frame}{Criptografia baseada em polinômios multivariados}
	\begin{align*}
	p^{(1)}(x_1,\dots,x_n)&= \sum_{i=1}^n \sum_{j=i}^n \alpha^{(1)}_{ij} x_i x_j
		+ \sum_{i=1}^n \beta^{(1)}_i x_i + \gamma^{(1)} \\
	\onslide<2->{
	p^{(2)}(x_1,\dots,x_n)&= \sum_{i=1}^n \sum_{j=i}^n \alpha^{(2)}_{ij} x_i x_j
		+ \sum_{i=1}^n \beta^{(2)}_i x_i + \gamma^{(2)} \\
	&\vdots \\
	p^{(m)}(x_1,\dots,x_n)&= \sum_{i=1}^n \sum_{j=i}^n \alpha^{(m)}_{ij} x_i x_j
		+ \sum_{i=1}^n \beta^{(m)}_i x_i + \gamma^{(m)}
	}
	\end{align*}
	\begin{itemize}
		\item<3-> Resolver sistemas de polinômios não lineares é
		NP-Hard~\cite{garey1979npc}.
		\item<4-> Podem ser usados como mapas.
		\item<5-> Pares de chaves muito grandes!
	\end{itemize}
\end{frame}

\begin{frame}{Construção Bipolar}
	\begin{figure}
		\centering
		\begin{tikzpicture}
			\node (h) at (0, 0) {$h \in \mathbb{F}^m$};
			\node (y) at (3, 0) {$y \in \mathbb{F}^m$};
			\node (x) at (6, 0) {$x \in \mathbb{F}^n$};
			\node (z) at (9, 0) {$\sigma \in \mathbb{F}^n$};
			\draw[-latex] (h) -- node[above]{$\mathcal{S}^{-1}$} (y);
			\draw[-latex] (y) --
			node[above]{$\mathcal{F}^{-1}$} node[below]{\textbf{Assinar}} (x);
			\draw[-latex] (x) -- node[above]{$\mathcal{T}^{-1}$} (z);
			\draw[-latex] (z) to[bend left=35] node[above]{$\mathcal{P}$}
			node[below]{\textbf{Verificar}} (h);
		\end{tikzpicture}
	\end{figure}
	\begin{itemize}
		\item<2-> $\mathcal{F}$ é um sistema de polinômios multivariados.
		\item<3-> $\mathcal{S}$ e $\mathcal{T}$ são mapas lineares, servem para
		esconder o mapa central.
		\item<4-> A chave pública é dada por $\mathcal{P} = \mathcal{S} \circ
		\mathcal{F} \circ \mathcal{T}$
		\item<5-> Pode ser utilizada para construir esquemas de assinatura
		digital.
		\item<6-> É a construção básica do Rainbow.
	\end{itemize}
\end{frame}

\begin{frame}{O princípio Óleo e Vinagre~\cite{patarin1997ov}}
	\begin{itemize}[<+->]
		\item Consiste em criar um sistema com dois conjuntos de variáveis.
		\item \begin{equation*}\label{eq:ovpolynomial}
			y^{(k)} =
			\underbrace{
				\sum_{i=1}^{v}\sum_{j=i}^{v} \alpha^{(k)}_{ij} x_i x_j}_{
			V \times V} +
			\underbrace{
				\sum_{i=v+1}^{v+o}\sum_{j=1}^{v} \beta^{(k)}_{ij} x_i x_j}_{
			O \times V} +
			\underbrace{
				\sum_{i=1}^{o+v} \gamma^{(k)}_{i} x_i}_{\text{linear}} +
			\underbrace{\delta^{(k)}}_{\text{constante}}
		\end{equation*}
		\item O sistema é linear em Óleo.
		\item Um sistema com ``$o$'' equações e ``$o$'' variáveis de óleo pode
		ser solucionado se valores forem selecionados para as variáveis de
		vinagre.
		\item A estrutura do sistema é escondida por um mapeamento das
		variáveis $\mathcal{T}$.
	\end{itemize}
\end{frame}

\begin{frame}{O esquema de assinatura digital Rainbow~\cite{ding2005rainbow}}
	\begin{itemize}[<+->]
		\item Consiste em criar múltiplas camadas de sistemas que tem a
		propriedade Óleo e Vinagre.
		\item Resolvendo uma camada, possibilita resolver a próxima, pois as
		variáveis de uma camada são as variáveis de vinagre da próxima.
		\item Foi submetido ao processo de padronização do
		NIST~\cite{ding2017nist}.
		\item Chaves de até 1.6MB!
	\end{itemize}
\end{frame}

\begin{frame}{Reusando variáveis de vinagre para reduzir a chave privada}
	\begin{itemize}[<+->]
		\item A escolha das variáveis de vinagre é necessária para linearizar a
		primeira camada.
		\item Podemos guardar apenas o sistema simplificado após a substituição
		das variáveis de vinagre.
		\item Existe a possibilidade de alguma das camadas não ser inversível.
		\item Dois métodos foram propostos para gerar novamente o mapa central.
		\item Um terceiro método usa a variante utilizada na submissão para o
		NIST~\cite{ding2017nist} e evita que o mapa central seja gerado
		novamente.
	\end{itemize}
\end{frame}

\begin{frame}{Reusando variáveis de vinagre para reduzir a chave privada}
	\begin{itemize}[<+->]
		\item Resultados publicados em~\cite{zambonin2019handling}.
		\begin{table}
		\begin{tabular}{|c|c|c|c|c|}
			\hline
			% header
			\thead{Segurança\\(bits)} &
			\thead{Parâmetros\\$(K, o_1, v_1, v_2)$} &
			\thead{$K_{pr}$\\(bytes)} & \thead{$K^{\eta}_{pr}$\\(bytes)} &
			\thead{Redução} \\ \hline
			% data
			80  & $(\mathbb{F}_{256},17,17,9)$  & 19208   & 5914   & -69.21\%
			\\ \hline
			100 & $(\mathbb{F}_{256},26,22,11)$ & 45450   & 11013  & -75.77\%
			\\ \hline
			128 & $(\mathbb{F}_{256},36,28,15)$ & 103704  & 22110  & -78.68\%
			\\ \hline
			192 & $(\mathbb{F}_{256},63,46,22)$ & 440638  & 71773  & -83.71\%
			\\ \hline
			256 & $(\mathbb{F}_{256},85,63,30)$ & 1086971 & 164721 & -84.85\%
			\\ \hline
		\end{tabular}
		\end{table}
		\item Implementação de uma prova de conceito em
		\url{https://github.com/matheuspb/rainbow-eta}
		\item Este método pode ser combinado ainda com outros métodos que
		reduzem a chave pública para atingir 71\% de redução no par de chaves.
	\end{itemize}
\end{frame}

\begin{frame}{Conclusão}
	\begin{itemize}[<+->]
		\item Foi proposto um método para reduzir o mapa central do Rainbow,
		sem a adição de estrutura.
		\item Esse método pode ser usado em combinação com outros métodos que
		reduzem o mapa público.
		\item Trabalhos futuros podem extender a análise de segurança do método
		proposto.
		\item Podem também abordar a questão dos \textit{side-channel} attacks.
	\end{itemize}
\end{frame}

\begin{frame}{Conclusão}
	\center{\Huge Obrigado!}
\end{frame}

\begin{frame}[allowframebreaks]{Referências}
	\bibliographystyle{apalike}
	\bibliography{ref}
\end{frame}

\end{document}
